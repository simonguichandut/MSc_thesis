\documentclass[../main.tex]{subfiles}

\def\-{\raisebox{.75pt}{-}} % shorter minus sign : \-


\begin{document}
\section{Expanded envelopes}

% \begin{itemize}
%     \item Equations for general EOS. Degenerate EOS case.
%     \item Numerical method
%     \item Results
% \end{itemize}

%At luminosities close to the Eddington luminosity, the atmosphere of the NS will be subjected to a lower effective gravity and will thus tend to expand hydrostatically.

For the same neutron star of mass $M=1.4 M_\odot$ and radius $R=12$ km, we now search for solutions to the equations of structure with $v=0$. The resulting profiles will represent expanded static envelopes where hydrostatic balance is sustained by the luminosity. As explained in the introduction, the local luminosity is a function of redshift (Eq.~\ref{eq:Linf}), with the luminosity at infinity $L
^\infty$ being a constant of each model.  The other free parameter which will determine the model is the photospheric radius, which we will define in the same way as for the wind models. The boundary condition at the surface will also be the same, so that we will be able to link burning layer conditions to either regime.

\subsection{Numerical integration  and boundary conditions}

\subsection{Results}

\biblio
\end{document}
