\documentclass[../main.tex]{subfiles}
\begin{document}
\thispagestyle{empty}
\chapter*{Abstract}
\addcontentsline{toc}{chapter}{Abstract}
Type I X-ray bursts are thermonuclear runaway events occuring on the surface of accreting neutron stars that result in bright X-ray flashes. These events can produce luminosities so high that hydrostatic balance is lifted and the star's photosphere expands drastically, from tens to thousands of kilometres. Using steady-state equations of general relativistic radiation hydrodynamics, we calculate solutions of expanded, hydrostatic, envelopes as well as super-Eddington winds. We construct a grid of models which can be used to interpret X-ray observations of bursts, and as an outer boundary for time-dependent simulations of the neutron star burning layer. Our results show that observations of small photospheres likely point to static envelopes rather than winds, that the neutron star radius can easily be over-estimated when deduced from burst spectra, and that steady-state models in general are not applicable in the early stages of the burst. The theoretical framework that we derive and numerical methods that we propose can also be used in other astrophysical applications, such as classical novae.

% \begingroup
% \let\clearpage\relax
\chapter*{Abrégé}
\addcontentsline{toc}{chapter}{Abrégé}
Les sursauts rayons X de type I sont des réactions thermonucléaires prenant place sur la surface d'étoiles à neutrons produisant de puissants éclats lumineux. Ces événements peuvent produire des luminosités si grandes que l'équilibre hydrostatique à la surface de l'étoile est rompu, provoquant une expansion de sa photosphère jusqu'à des dizaines, voire des milliers de kilomètres. En utilisant des équations stationnaires d'hydrodynamique radiative relativiste, nous calculons des solutions d'enveloppes hydrostatiques étendues et de vents super-Eddington. Nous construisons une grille de modèles qui peuvent être utilisés pour interpréter des observations de sursauts rayons X, ainsi qu'en tant que condition frontière pour des simulations de l'évolution temporelle de la couche enflammée de l'étoile à neutrons. Nos résultats démontrent que les observations de petites photosphères suggèrent la présence d'enveloppes statiques plutôt que de vents, que le rayon de l'étoile peut facilement être surestimé lorsque déterminé à partir du spectre des sursauts rayons X, et que les modèles stationnaires en général ne sont pas appropriés pour décrire l'évolution initiale du sursaut. Le cadre théorique que nous dérivons et les méthodes numériques que nous proposons peuvent également être utilisés pour d'autres problèmes en astrophysique, tels que les nov{\ae}.

% \endgroup

\end{document}
